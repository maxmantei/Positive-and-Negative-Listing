\documentclass{article}
\usepackage[utf8]{inputenc}
\usepackage{setspace}
\usepackage[a4paper, top=2cm, left=3cm, right=3cm, bottom=2cm]{geometry}
\usepackage{url}
\usepackage{hyperref}
\hypersetup{
    colorlinks=true,
    linkcolor=blue,
    urlcolor=blue
}
\usepackage{amsmath}




\title{WORKING PAPER\\Is the glass half empty or half full?\\Rethinking Negative and Positive Listing in Trade in Services Agreements}

\author{
  Adriana Espés\\
  \textit{ILE Graduate School, University of Hamburg}\\
  \url{adriana.espespizarro@ile-graduateschool.de}
  \and
  LastName2, FirstName2\\
  \texttt{first2.last2@xxxxx.com}
}

\date{\today}


\usepackage{natbib}
\usepackage{graphicx}

\begin{document}

\onehalfspacing

\maketitle

\begin{abstract}
    Nowadays, services are the fastest growing sector of the global economy and represent nearly 20\% of global trade. However, research on trade in services is still in its infancy when compared with trade in goods. This paper focuses on an essential part of trade in services regulation: the two main methods adopted by countries to formulate and organize access to their services markets — in other words, the positive and the negative listing. Both scheduling strategies have been used among Regional Trade Agreements since the General Agreement on Trade in Services (GATS) and the North American Free Trade Agreement (NAFTA) created, respectively, these two opposed styles. The paper offers a better understanding of the trade liberalization consequences of both scheduling strategies, and provides an extensive compilation of the different characteristics discerned by scholars. Finally, it compares both scheduling approaches through a unique econometric evaluation based on data from 50 different Regional Trade in Services Agreements. Results show for the first time that the negative list is correlated with higher degrees of liberalization between partner countries. Therefore, this paper also provides evidence for the thesis that default choices matter.
\end{abstract}

\doublespacing

\section{Introduction}
Trade in services represents the fastest growing sector of the global economy and accounts for two thirds of global output\footnote{WTO, 2017}. However, even if there has been vast empirical literature on trade in goods, trade in services research is still a recent development. When it comes to analyzing possible consequences of trade in services’ regulation, the lack of investigation is even more evident.

\smallskip

Until the 1990s, regional trade agreements (from now on, RTAs) were not numerous and limited in scope, with almost no mention of trade in services. With the arrival of the General Agreement on Trade in Services (GATS) and the North American Free Trade Agreement (NAFTA) in 1995, the landscape for RTAs changed completely. Nowadays, according to the World Trade Organization\footnote{WTO’s Regional Trade Agreements Information System (RTA-IS). Accessed March 2018.}, 151 RTAs with a services component are in force.

\smallskip

GATS and NAFTA also meant a significant cornerstone for trade in services regulation. Both treaties created their own singular way of portraying trade in services commitments. The subsequent RTAs followed their steps in how their participant countries depicted their trade in services schedules – as reservations or concessions. In trade in services agreements, the same liberalization commitment can be formulated in either positive (related with the GATS-style) or negative (related with the NAFTA-style) way. In the positive list approach, no commitments to liberalize are undertaken by a state unless explicitly inscribed in a schedule of commitments, and to the extent of the inscription. In the negative list approach, all services sectors and disciplines are fully committed to free trade unless expressly excluded, to the extent of the exclusion. The aim of this paper is to analyze the divergence between the positive list and the negative list approaches and their consequences for RTAs liberalization level. It contributes to the discussion of whether both approaches reach the same level of openness. So far, the opinions of researchers are controversial, with almost no empirical research done to date. Only one previous paper (Fink and Molinuevo 2007) tried an econometric investigation that attempted to test if negative list agreements are effectively associated with wider and deeper liberalization. However, this study relied on a basic model and was geographically restrained. This paper tries to enhance Fink and Molinuevo's initial model and, therefore, gives additional insight to the theoretical discussion.

\smallskip

The paper proceeds as follows. Section 2 describes the origins of both GATS and NAFTA agreements and how they are related with the positive and negative list approach. It ends with a regulatory outlook of different scheduling techniques in RTAs. Section 3 depicts the diverse opinions of scholars on the scheduling debate. It compiles all the arguments on why the negative list approach reaches higher liberalization levels – and how the positive list could counteract them. Section 4 describes the empirical strategy, it includes a detailed analysis of the dependent variable, an index created by Miroudot et al. (2011) which studies the liberalization degree of RTA’s commitments. 50 bilateral RTAs’ schedules commitments are analyzed to see if the negative list is correlated with higher treaties’ level of openness. 25 of these treaties follow a negative list approach. The other 25 are positive list schedules. Results of these econometric estimations, using a fractional regression model, are presented in Section 5. It is found strong support for the hypothesis that negative list agreements determine higher and deeper liberalization levels. Finally, Section 6 concludes.

\section{Background -- history and landscape of scheduling approaches}

The regulation of trade in services is a relatively recent phenomenon when compared with trade in goods. The concept of trade in goods is a relatively simple idea (a product is transported from one country to another) when compared with trade in services, a concept much more diverse (WTO, 2017a). Although inside the General Agreement on Trade in Services (GATS) a significant number of terms and notions have been borrowed from the General Agreement on Tariffs and Trade (GATT), there are also relevant differences between them (Adlung and Mattoo, 2008: 48). These differences help to explain why regulation of trade in services is slightly more difficult and convoluted than the classical trade in goods measures. First of all, the definition of trade in services extends beyond the notion of cross-border exchange, it also covers consumer movements and factor flows (investment and labor). The scope of relevant disciplines is not only confined to the treatment of products (the service per se), but also extends to measures affecting its consumers and suppliers (services producers, traders and distributors). The coverage is then extended and trade in services regulation presents a flexible structure. Unlike under the GATT, the use of quantitative restrictions and denials of national treatment are not prohibited and they could be subject of negotiable commitments. The application of these commitments to services sectors is defined in country-specific schedules, in which no common templates exist – it is impossible to find two identical schedules (Adlung and Mattoo 2008: 48).

\smallskip

As Fink and Molinuevo (2007: 8) explain, trade agreements’ main objective is to promote international commerce. They can do that in three ways: by reducing barriers to foreign participation, by making trade policies more transparent and by improving the credibility of a trade regime. The different architectural choices of an agreement can make an important difference in this triple way of promoting international commerce. One key point of the architectural choice and main characteristic of an agreement’s legal design is the so called scheduling of commitments. Schedules are legal instruments that guarantee the specified access conditions (Adlung and Mattoo 2008: 56). While in trade in goods schedules of commitments consist of product concessions mostly based on tariffs with a harmonized system to classify the products, trade in services compiles broad measure coverage with no uniform classification of services sectors. These measures affecting trade in services could take virtually any form, including that of law, regulation, rule, procedure, decision or administrative action (Article XXVIII.a GATS). It can be seen why, compared to traditional tariff negotiations under trade in goods, services negotiations tend to require more intense domestic consultation and coordination (Adlung and Roy 2005: 5). This article focuses on a special feature of trade in services schedules of commitments, which actually could make a difference in the level of openness reached.

\smallskip

There are two major approaches towards the design of schedules of commitments, two major approaches towards the liberalization of trade and investment in services which are present in the WTO and RTAs (Mattoo and Sauvé 2008: 253). They are the positive list, also called "bottom-up" approach and related to the GATS-style; and the negative list, or "top down/list it or lose it approach" and related to the NAFTA-style.

\smallskip

With the positive list, no commitments to liberalize are undertaken by a state unless explicitly inscribed in their schedule of commitments (Broude and Moses 2015: 385). In other words, the members to an agreement list their commitments specifying the type of conditions under which foreign services suppliers can enter a given market and also the type of treatment that will be granted. In this approach, the specific commitments may be modified and withdrawn after a certain period of time. Liberalization is progressively achieved through rounds of negotiations among members to an agreement. The commitments are undertaken for each service sector, sub-sector or activity and, once listed, are considered to be binding (Nikomborirak and Stephenson 2001: 2). It is usually associated with the method used by the GATS in specific commitments and in subsequent negotiations at the Doha Round. However, as it will be shown later, this association is sometimes misleading.

\smallskip

With the negative list, all services sectors and disciplines are fully committed unless expressly excluded, to the extent of the exclusion (Broude and Moses 2015: 385). In this approach all service sectors and measures are to be liberalized unless otherwise specified in the agreements' annexes containing reservations, also called nonconforming measures. Any exceptions to sectoral coverage and to non-discriminatory treatment must be specified in these annexes. The non-conforming measures in the annexes may either be in the form of permanent exceptions or may be subject to future liberalizations (Nikomborirak and Stephenson 2001: 2). This approach is often associated with the model used by the NAFTA.

\smallskip

Most of the RTAs follow one of these approaches. Since positive listing and negative listing are extremely related to the GATS and NAFTA styles, both treaty origins and features are explained hereafter. It is relevant to know the characteristics of both treaties in order to understand how the two approaches to trade in services’ scheduling work.

\subsection{The GATS-style}

Before the GATS entered into force in 1995, the lack of a multilateral legal framework was no obstacle to the increase of services trade flows. Back then, trade in services was regulated through bilateral and regional schemes (Marchetti and Mavroidis, 2011: 690). Apart from industry-specific cooperation and some US free trade agreements, the EU was the only supranational entity with experience in regional liberalization of trade in services (Marchetti and Mavroidis, 2011: 691). After debates over the inclusion of Trade in Services in the GATT negotiations, the GATT contracting parties decided through a Ministerial Declaration\footnote{Statement by the Chairman and Adoption of the Ministerial Declaration on the Uruguay Round, MIN.DEC, 20 September 1986, at Punta del Este.} to finally include negotiations on trade in services, which launched the Uruguay Round (1986-1994). It was at this multilateral negotiating framework where the parties also decided to establish a separate agreement on services, without repercussions on the trade in goods negotiations (Fuchs, 2008: 5), and GATS was born. At the beginning of the negotiations of trade in services, in October 1989, the US and the European Community proposed to establish the negative list approach as a way to cover sectors in the GATS embryo. Both parties envisaged the future agreement to cover all services except all those explicitly excluded by the proposed negative list. Meanwhile, most of the developing countries supported a positive list approach, meaning that all sectors included in the treaty would be the ones covered (Fuchs, 2008: 8). Therefore, there was a conflict from the very beginning between how to portray the sectors at the treaty-to-be. GATS finally entered into force in January 1995, with all WTO members assuming commitments in service sectors in diverse degrees (WTO, 2017b).

\smallskip

GATS created a set of rules that were in some aspects inspired by previous concepts defined by GATT. Other concepts of GATS were newly developed and became a reference for trade in services agreements. GATS rules operate at two levels that define the extent of liberalization undertaken by individual countries. Some rules apply across all the measures affecting trade in services, others only apply to sector-specific commitments.

\smallskip

Within the first group, the most important general rules are transparency and the most-favored-nation (MFN) principle (Mattoo and Stern 2008: 25). The former requires that all measures of general application affecting trade in services from a Member are published, and that the rest of Members are informed in those measures’ changes. The latter, a concept borrowed from GATT, prevents Members from discriminating between trading partners (Mattoo and Stern 2008: 25).

\smallskip

On the other hand, sector-specific commitments are divided between modes of supply and rules about market access and national treatment. Modes of supply are a signature feature of GATS and constitute the definition of trade in services in its Article I. The four modes are cross-border supply (or mode 1, from the territory of one Member into the territory of any other Member), consumption abroad (or mode 2, in the territory of one Member to the service consumer of any other Member), commercial presence (or mode 3, by a service supplier of one Member, through commercial presence in the territory of any other Member) and presence of natural persons (or mode 4, by a service supplier of one Member, through presence of natural persons of a Member in the territory of any other Member). Commitments in these four modes are then divided in two main disciplines of rules, which constraint governments from adversely restraining foreign services suppliers’ participation in the domestic economy: market access and national treatment. Market access (article XVI) is defined by an exhaustive catalogue of explicit trade barriers. The provision encompasses four types of quantitative restrictions, limitations on the form of legal establishment and restrictions on foreign equity participation. Measures covered may be discriminatory or non-discriminatory in nature (Fink and Molinuevo 2007: 22). Unlike article XVI, national treatment (article XVII) does not provide an exhaustive list of measures. It is just the mandate that imported services do not face more restrictive policy measures than domestically supplied services once they act inside the domestic market. Article XVII.2 makes clear that limitations on national treatment cover cases of both de jure and de facto discrimination. GATS article XX.2 addresses the overlap between market access and national treatment measures, and specifies that measures inconsistent with both disciplines are to be scheduled under market access and would then be considered as a limitation on national treatment as well.

\smallskip

Finally, after the initial debate portrayed before, the definitive listing of this commitments adopted a so-called "hybrid approach", showing a convergence of interest between the negotiating parties. This hybrid strategy combines elements of both positive and negative listing. GATS features a positive list of sectors, whereby countries need to expressly state the services sectors subject to the market access and national treatment obligations. Contrary to the proposed negative list, obligations in the GATS do not apply to all services, but just to the extent of those activities that parties have inscribed in their schedules of commitments (Molinuevo 2008: 455). In the treaty the information is entered by columns. There is a main column which clearly defines the sector, subsector or activity that is the subject of the specific commitment. Members are free to identify which sector they will list in their schedules, and it is only to this extent that the commitments apply. The classification of sectors was also free to choose, but in the majority of schedules the sectors’ order follow the GATT Secretariat classification, which lists twelve broad sectors. In most cases, the sectorial entries are accompanied by numerical references to the Central Product Classification system of the United Nations, which gives a detailed explanation of the activities covered by each sector or subsector (WTO, 2017c).

\smallskip

Under those sectors, GATS also granted flexibility to the parties to undertake commitments in the scheduled sector at any degree of openness. Countries could undertake full commitments, meaning they commit to apply no measures denying either market access or national treatment to the relevant sector and concrete mode of supply. This is inscribed by the term "None" (i.e. "no limitations", full commitment). On the other hand, countries could also exclude a given mode of supply from the scope of the market access and/or national treatment provisions. This lack of commitment in a scheduled sector is indicated by the entry "Unbound". When all modes of supply are unbound for both market access and national treatment obligations, the sector should not be listed in the schedule of commitments and is fully closed for that country. Between the "None" and the "Unbound" options, countries may undertake partial commitments or, in other words, commitments with certain limitations for a given mode of supply and provision. In these options, Members indicate the restriction they wish to maintain, which could be reflected by the terms "None, except [$\ldots$]". This described mechanism of inscribing the trade-restrictive measures does not perfectly match the positive or negative list approach. As we said, in a positive list approach countries are required to describe the measures that will be applied in conformity with the agreements’ obligations, but no commitments are undertaken with regard to non-listed measures. While within the negative list only the measures that do not conform to the agreement’s disciplines are the ones featured. Therefore, the GATS scheduled of commitments cannot be described as a pure positive list. The services sectors are effectively listed in a positive list way, but once inside the measures can be inscribed in either a positive or a negative manner (Molinuevo 2008: 455). This is why the GATS-style should be differentiated from a "pure" positive list. Here, Members take a two-step decision. First, they decide which services sectors are subject to market access and national treatment disciplines. Then, they decide which measures violating market access and/or national treatment will be kept in place for each mode in that sector (Mattoo and Stern 2008: 26). In the latter choice, parties may define the level of openness in listed sectors in either on a positive or negative way. This is why GATS is usually defined as a hybrid list. As stated before, the GATS-style for the scheduling of commitments has frequently been referred to in the literature as a positive list approach. To better define the concept, some scholars state that this terminology should only be used to refer the selection of sectors subject to trade commitments (Fink and Molinuevo 2007: 12). For the purpose of this research, positive list agreements are considered the ones whose default rule is no liberalization. As Broude and Moses explain, GATS overall starting point for negotiations was originally of zero commitment and parties subsequently increase in a gradual way their level of openness with every commitment. Therefore, GATS general framing is positive (2016: 392).

\smallskip

To sum up, GATS started a specific regulation, adapted to trade in services and changed the landscape for numerous RTAs. GATS Article V promoted the creation of new trade in services agreements among its members, and it could be seen that most of the positive list agreements have a strong GATS influence.

\subsection{The NAFTA-style}

At almost the same time as GATS, NAFTA came into effect. In 1990, Canada, the U.S. and Mexico started the negotiations of a free trade agreement that entered into force the 1st of January 1994. The fifth part of NAFTA established principles on different chapters about investment, cross-trade in services, financial services, telecommunications and temporary entry of business persons. Apart from this division on chapters, the principles about cross-trade services were comparable to the GATS definition, being applied to the provision of services from one member to another, also within a NAFTA country by an individual provider from another member and consumption of services in one NAFTA country by consumers from another one. However, NAFTA contains Annexes where each party is allowed to maintain measures which do not conform to the previously described principles of the chapters -- in other words, the reservations schedules of Mexico, North America and Canada.

\smallskip

With NAFTA another trade in services agreement design style was introduced, and it differs significantly from GATS in many respects. The main difference highlighted by scholars is that NAFTA is based on a negative list scheduling modality (Roy et al 2006, Terry 2010). By NAFTA everything is liberalized unless otherwise indicated through the list of reservations. These reservations are divided between existing non-confirming measures (Annex 1) and future measures (Annex 2). Therefore, negative list agreements usually follow this model based on different annexes dividing the time extent of the measures. As will be more thoroughly explained later, these agreements provide a high degree of transparency since the actual level of openness is spelled out, giving an indication of the regulatory framework of the members in place (Roy et al 2006: 9). However, as it has been stated, in the NAFTA-style matters relating to trade in services are regulated in separate chapters: cross-border trade in services, investment, financial services, telecommunications and temporary entry of business persons. It is then essential to fully comprehend the scope of each chapter and the manner in which the different chapters interrelate, in order to accurately understand the implications of the specific commitments.

\smallskip

Therefore, NAFTA-style differs in how the different modes of supply from GATS are dealt with in different chapters. Modes 1, 2 and 4 would be part of a chapter on cross-border trade in services and disciplines related to mode 3 would be a part of a chapter on investment for services and non-services activities (Roy et al. 2006: 9). This is what Lee and Latrille (2012: 9) called the “3+1” approach. It might seem that covering the modes of supply divided in chapters would make no meaningful difference regarding the obligations. However, the main disciplines seen at the GATS-style also changed at NAFTA. While both cross-border services and investment chapters each contain a national treatment obligation, neither contains the GATS article XVI market access obligation for certain non-discriminatory quantitative restrictions (Roy et al 2006: 9). Albeit NAFTA’s cross-border services chapter contains a provision on non-discriminatory quantitative restrictions, it is just a best endeavor to notify and negotiate down quantitative restrictions, which are in fact defined in similar terms to those of article XVI (Lee and Latrille 2012: 19). The investment chapter, for commercial presence in services, does not directly include disciplines on non-discriminatory quantitative restrictions. In this sense, GATS went further (Roy et al 2006: 9). Actually, agreements influenced by NAFTA-style usually change this feature and incorporate a binding market clause, which is identical to that of article XVI of GATS except for its last restriction on limitation of foreign ownership, and they also extend the scope of this services market access clause to the investment chapter (Lee and Latrille 2012: 9).

\smallskip

Regarding specific sectoral rules, it is difficult for NAFTA to operate as GATS does due to the negative list nature. For example, GATS included three sectoral annexes on air transport, telecommunication and financial services, which develop and complement normal GATS rules. Also, GATS includes an exception for services delivered in the exercise of governmental authority. Since NAFTA is a negative list agreement, all sectors are understood to be included, even services and investment related with state actions, except to the extent specifically carved out through reservations or exceptions (VanDuzer 2014: 3). When drafting the exceptions or reservations, their sector, sub-sector and industry classification has to be included as a specific element. NAFTA-style also leaves free hand to the parties to classify those sectors at the reservation’s description. For example, the NAFTA Annex also used the Central Product Classification numbers, except Mexico, that used their own product and activity classification (\textit{Clasificación Mexicana de Actividades y Productos}, CMAP).

\smallskip

These are the main differences between the NAFTA-style and the GATS-style agreements at an architectural level. Both styles differ in more key areas, like the treatment of domestic regulation, rules of origin, institutional provisions, etc. The choice of the differences analyzed above stems for their importance when differentiating the negative and the positive listing origins and characteristics. It is also important to know these differences in order to understand the econometric strategy that will be used afterwards.

\subsection{The scheduling approach classification of Regional Trade Agreements}

As of March 2018, there were 151 RTAs with a services component in force registered by the WTO\footnote{WTO’s Regional Trade Agreements Information System (RTA-IS). Accessed March 2018.}. Even if they are still less than the ones dedicated to trade in goods, their growth is recent and exponential. Most of these treaties could be grouped at the previously described two styles of agreements: those based on a positive list GATS-style structure and those with a negative list approach based on the NAFTA. For example, there are 50 RTAs analyzed in this paper, and they are equally divided between the positive and negative list approach.\footnote{Table 1 at the Appendix classifies them.}

\smallskip

This division of styles is difficult in geographical terms. It seems that the intra-American agreements (North and Latin America) tend to choose the negative list as their scheduling design, perhaps due to the NAFTA influence. One significant exception is MERCOSUR, which follows accurately the GATS-style. The influences for the rest of the world are not that transparent. Lee and Latrille (2012: 9) state that it appears that developing countries are more prone to positive listing, except for the Latin American region. Also, there are some countries that cannot be related with one style since they change it depending on their trading partner. For instance, Japan’s Economic Partnership Agreements (EPAs) with Indonesia, Malaysia, the Philippines, and Thailand have been conducted in a positive list, while those with Chile and Switzerland take a negative list approach\footnote{All of them have been included in the econometric analysis.} (Sauvé and Mattoo 2011: 254).

\smallskip

Moreover, the classification of RTAs is also complicated when they increasingly mix positive and negative list approaches under the same agreement. For example, the negative list is widely used in the investment area; still some agreements decide to combine it with positive listing for the rest of the modes or the cross-border trade chapter. Others use at the same agreement negative listing for banking services and positive listing for insurance services (Sauvé and Mattoo 2011: 254). As a curious example, in the recent China-Australia Free Trade Agreement (ChAFTA, December 2015) each party decided to choose a different scheduling approach and listed it together at the agreement’s annexes. Australia defined its reservations following a negative list approach, while China undertook its commitments under the GATS-style.\footnote{Inside the treaties analysed, we can find the case of Korea-Singapore FTA, a negative list agreement except for the financial services sector.}

\smallskip

However, sometimes the clarity of this two-type classification is often questioned. As Broude and Moses (2016: 392) clarify, the line between positive and negative listing is rarely a bright one. As stated before, there are even doubts about how to define the GATS model (positive or hybrid approach). Furthermore, there is lately an increasing number of RTAs which choose a hybrid approach. They mix in the same country’s schedules of commitments characteristics that are related to one of the previously described styles. A relevant case is the future Trade in Services Agreement (TiSA), a multilateral agreement currently negotiated among some WTO members, whose main purpose is to be the new GATS. TiSA’s architecture is based on GATS; however, the national treatment commitments would be scheduled on a negative list approach, while market access follows the positive list approach. The treaty contains a horizontal national treatment principle in its main text, which would apply to all sectors unless parties register an exception or limitation (European Commission, 2017). Meanwhile, some internal EU treaties cannot fit NAFTA nor GATS, since these treaties’ objectives are deeper economic and political integration and have a longer history. The common features of these EU agreements are a “neither nor” approach to scheduling of commitments, with no modes and the use of alternative concepts proposed as principles like “freedom to provide services and establishment” (Lee and Latrille 2012: 9). Another example of outliers is provided by Fink and Molinuevo (2007), in one of the most extensive classifications of different scheduling approaches between 25 East Asian FTAs. The authors found GATS-style and negative lists agreements in their sample, but they also found two cases of pure positive list agreements. The Mainland-Hong Kong and the Mainland-Macao Closer Economic Partnership Arrangements do not establish binding disciplines, like the Market Access and National Treatment provisions created by the GATS, and they also do not define any modes of supply (2007: 12). The parties just enumerate their commitments on this pure positive list approach, specifying for each listed service sector the level and type of foreign participation that is allowed.

\smallskip

As it can be seen, the classification of all the RTAs is not straightforward. For the purpose of this article and in order to avoid misunderstandings, the division between RTAs’ schedules of commitments is done at a country level. Once there, it is considered what happens at the zero-commitment level under the country schedules at issue. For example, the schedules are classified as a negative list if everything is liberalized, unless the parties gradually specify the contrary. And vice versa.

%% TWO TABLES HERE

\section{Analyzing positive versus negative listing. Nothing negative about negative listing?}

As explained earlier, the same commitment can be formulated in either positive or negative terms. However, it is not clear if both approaches achieve different levels of liberalization. So far, the opinions are controversial. Some authors claim that a decision on the top down or bottom up approach will determine the liberalizing modality (Nikomborirak and Stephenson 2001: 21). On the other hand, others assert that from a primary level this question can be viewed as trivial, since it only concerns the choice between saying what is to be done and what is not to be done (Low and Mattoo 2000: 22). In order to discern the advantages claimed by the negative list advocates, the following paragraphs introduce the main points claimed by researchers so far.

\subsection{Transparency}

First, even the scholars who sustain that both approaches might lead to the same level of liberalization usually recognize that the negative list approach is associated with a higher level of transparency. For example, Sauvé and Hoekman stated that “while either approach can lead to the same liberalization outcome, a negative list is significantly more transparent because it forces parties to reveal all non-conforming measures and excluded sectors” (1994: 10). The authors sustain that one of the main GATS-style architectural differences is that it allows hiding the sectors where the Members want to continue with discriminatory practices. At the end, negative listing would not allow to exclude the coverage of sensitive industries. Stephenson (2002) also defended the higher transparency of the negative list, since the annexes of negative list agreements usually provide a complete description of the existing restrictions to market access or national treatment in the Member’s market. If we have a look to the GATS-style agreements, reservations are classified by sector. However, at the negative list approach, such clauses are listed based on measures at the level of regulatory practice, divided between discriminatory measures and non-discriminatory measures. (Stephenson 2002: 194). This plays a key role for transparency in architectural choices, since it makes easier for foreigner providers to discern in which way they are affected.

\smallskip

But this superiority of the negative list always comes with considerations. As Fink and Molinuevo (2007: 17) enumerate, the transparency value of the negative list depends on the level of openness. To detect what is not allowed might be more difficult and exhausting when the list becomes long. When the trade-restrictive measures are the size of a “telephone book”, the transparency characteristic is distorted, making difficult to understand for foreign businesses what is allowed. Like always, transparency will be a matter of the parties will and the nature of the non-confirming measures scheduled. If these measures do not reflect the laws and regulations from the member country, services providers cannot figure out an accurate picture of the level of openness in that economy (Fink and Molinuevo 2007: 18). Furthermore, some negative list agreements might allow open-ended exclusions that dilute the transparency virtue. The NAFTA, paradigm of the negative list, includes in Mexico’s and United States’ schedules their right to adopt or maintain any restrictive measure relating to investment in specific sensitive sectors (telecommunications for Mexico and maritime transportations services for US\footnote{See World Bank (2004) for further specification}). If a negative list approach includes this kind of open-ended exclusions, then the differences between the two discussed methods become irrelevant in this sense. Moreover, the transparency benefits could be matched by the positive list approach if the agreement forces all members to list all sectors (World Bank 2004: 240). Inside the negative listing approach, if members are not ready to make commitments in sensitive services sectors, they might be tempted to “specify heavy handed restricting measures in their negative lists” (WB 2004: 241), whereas in a positive list this will mean to entry that sector or even subsector as “unbound”. This, among other factors, might explain why some negative list RTAs have used positive listing in scheduling commitments for sectors where there is a regulatory sensitivity. Fink and Molinuevo (2007: 18) highlight this use for the financial sector in four East Asian negative list RTAs.

\subsection{Status quo}

Similar to the transparency argument, Fink and Molinuevo (2007: 7) state that non-conforming measures scheduled under a negative list reflect the country’s status quo policies. This also helps interested RTA’s foreigner providers, who know directly beforehand the laws and regulations affecting their rights to entry and to operate at the partner’s market. As they claim, this status quo binding also maximizes the credibility value of trade commitments, since the counterparts secure that the written policies would not become more restrictive. It is clear that if an RTA uses an opt out negative list approach instead of the positive one; the agreements’ exceptions are particularly important since the default assumption is that all services are covered (Terry 2010: 916). As Sauvé highlighted, the negative approach creates a standstill. These FTAs establish a baseline of liberalization by blocking the regulatory status quo. According to Sauvé, the GATS-style agreements could therefore generate a wedge between what it is really applied and what is bound by the regulation (2002: 19).

\smallskip

However, the benefits linked to the locking of regulatory status quo might be also achieved with positive list agreements. As Fink and Molinuevo (2007: 18) state, certain positive list agreements have a requirement to schedule at the level of existing policies. Leveling this pro-liberalization advantage from the negative list is as easy as to introduce such a clause in a positive or hybrid list agreement (the authors mention the examples of the Japan-Malaysia EPA and the Japan-Philippines EPA). Moreover, negative list agreements can also include commitments that are more restrictive than the status quo. The so-called ceiling bindings are members’ commitments that might offer less guaranteed access or legal security than the actual domestic policy of the country (World Bank 2004: 242).

\subsection{Future Developments}

Thinking also about the long-term benefits, a negative list approach implies that “any new services developed as a result of innovation or technological advancement, or for any other reason, would automatically be subject to established disciplines” (Low and Mattoo, 2000: 22). Lapid suggests that a negative list framework serves to prioritize parties’ interest and swiftly responds to economic and political changes. Once there are new forms of services, the parties’ won’t need to constantly update and modify the list, as they only would need to take action in regard to unliberalized sectors. According to the author, the negative listing approach serves better the WTO’s agenda and purpose: promoting liberalization as the default rule. Therefore, her view also defends the negative listing as the architecture approach which entails higher liberalization levels (2015: 28). Another long term virtue noted by Snape and Bosworth (1996:200), is that negative list allows better comprehensive and cross-country comparison of restriction levels. This provides the negotiators a clear inventory of what needs to be improved in the future.

\smallskip

However, in more recent articles, Sauvé (together with Mattoo, 2010b: 43 and 2011: 253) clarifies that some specific annexes of negative agreements allow the Parties to lodge reservations that preserve their future regulatory flexibility in sectors, subsectors and modes of supply. This would resemble the “unbound” or non-scheduled commitment under a GATS-style agreement. But in this case again negative listing promotes transparency, since by its nature, parties are forced to reveal the nature of that concrete existing non-conforming measure in such reserved sector. While in GATS, there is no information on the type of non-conforming measure in the sensitive sector.

\smallskip

It seems possible to replicate a negative list agreement with a positive list schedule. For example, as Fink and Molinuevo explain, the long term pro-liberalization benefits of the negative listing could be translated into a positive list. The bottom up list could apply to future service activities by covering residual sectors with commitments on “other business services” or “other services not included elsewhere”. Therefore, future sectors could be also liberalized through a positive list approach, even if the coverage for these residual categories would not be as substantial and unfettered as the one reached by the negative list (Fink and Molinuevo 2007: 17). The authors also found agreements designed following the negative listing that undermine the pro-liberalization scope for future services. For example, at the Japan-Mexico EPA, there is an Annex for Chapter 7 and 8 (Investment and Cross-Border Trade in Services) called “Reservations for Future Measures”. In this Annex both parties scheduled a measure under which they reserve “(…) the right to adopt or maintain any measure relating to new services other than those services recognized (…) at the time of entry into force of this Agreement”. Moreover, both parties reserve “(…) the right to adopt or maintain any measure relating to the supply of services in any mode of supply in which those services were not technically feasible at the time of entry of force of this Agreement” (Fink and Molinuevo 2007: 19). This is be another case of how a negative list agreement might also limit one of its main pro-liberalization advantages, behaving as cautious as a positive list approach.

\smallskip

At this point, it is important to add that introducing future nonconforming measures might also raise concerns. With a negative list governments renounce to introduce access-impairing measures for the future and non-existent sectors (Mattoo and Sauvé 2011: 253). This means that everything that is not regulated at the time of the agreement’s entry into force is supposed to be open market, without the possibility of introducing discriminatory measures. In the aftermath of new information and communications technologies, countries might want to control their future regulatory freedom through a positive list approach.

\subsection{Disclosure at the negotiating level}

Another advantage of the negative list can be observed at the negotiating level. Low and Mattoo (2000: 22) defend that a negative list might create a “pro-liberalization dynamic”, since some governments could be embarrassed by their long list of exceptions. Nevertheless they also realize this might happen at the positive list approach, when the list of commitments is a short one. Negotiating a negative list agreement is often described as administratively burdensome, particularly by developing countries. For example, many developing countries were only able to access the GATS with minimal commitments, since a negative list was too resource intensive for them; especially within a short time period at the completion of the round. This was a key point to adopt the positive list approach at the GATS schedules (Hoekman 1995: 28). Also Sauvé (1996: 10) acknowledges that a negative list might be administratively burdensome, particularly for developing countries. But he defends that “the gains in transparency and in user friendliness tend to outweigh such an administrative burden”. He proposed a higher level of progressivity when negotiating non-conforming measures to soften such an administrative burden. As can be inferred, in the process of negotiating, a negative list also forces governments to reveal most of their non-conforming measures. If they want to keep the restrictions they must defend their rationale (Fink and Molinuevo 2007: 17). Moreover, the negative list encourages parties to perform a comprehensive audit of their internal trade restrictive measures, which enhances the dialogue between trade negotiating and regulatory communities (Mattoo and Sauvé 2011: 252). This adds another transparency and governance-enhancing mechanism to the negative list before reflecting the commitments at the agreement.

\subsection{Sectorial limitation}

Scholars also worship the negative list’s classification of sectors. A criticism against the positive list is that commitments, especially market access ones, only apply to those sectors that the member chooses to list. The GATS-style creates a framework where members could remain free to restrict trade on one specific sector. Mattoo and Wunsch affirm that “this positive list approach places a heavy burden on the services classification scheme used by Members” (2004: 2). Hoekman explains that the positive list can create difficulties for a government to add sectors to its schedule. The affected industries might lobby their governments against the inclusion of their incumbent sector. Meanwhile, with a negative list approach the affected industries must seek why the commitments must apply to them. The industries lose their bargaining power when the rules apply to everyone and have multilateral scope, therefore they must justify and confront their government on why certain rules must protect them. (Hoekman 1995: 27).

\smallskip

On the other hand, practice shows a negative list approach can also introduce new non-conforming measures in sensitive sectors. A significant number of negative list agreements allow their parties to list sectors and activities in which regulatory immunity is preserved, which is again the equivalent of an unbound commitment at the GATS-style (UNCTAD, 2006: 20). This means that also relevant industries might exert their influence to keep their selves out of the scope of the agreement, even under negative list negotiations. Moreover, GATS-style agreements usually include horizontal commitments. Measures scheduled in horizontal commitments apply to all listed services sectors (Fink and Molinuevo 2007: 13). Even if they usually fix a low level of openness across sectors, this strategy will avoid the sectorial classification criticized by Mattoo and Wunsch.

\subsection{The ratchet mechanism}

Last but not least, another pro-liberalization characteristic of negative schedules is the so-called ratchet mechanism. This consists of a provision whereby any autonomous liberalization measure undertaken by a RTA member between negotiations becomes automatically part of that member’s schedule of commitments (Mattoo and Sauvé 2011: 252). They work as credibility-enhancing provisions, and usually it is a way to signal to foreign providers that there is a commitment not to reverse liberalization.

\smallskip

But again, ratchet clauses can also be introduced in a positive list agreement. Fink and Molinuevo (2007: 13) describe how Japan’s EPAs with Malaysia and the Philippines introduced this innovative clause to reduce uncertainty and increase binding effects. These GATS-style agreements include the chance to identify in the respective country’s schedules in which services sectors they want to bind status quo policies. In addition, these identified sectors are also subject to upward ratcheting. Once one of the countries decides to eliminate a trade-restrictive measure, the policy towards the other parties will automatically be bound at the more liberal level (Fink and Molinuevo 2007: 13). At the end, ratcheting is a mere clause which should not be linked to one of the listing strategies. However, like the majority of the promoting liberalization clauses, the ratchet effect is a mechanism which can deprive the parties of their regulatory flexibility (UNCTAD 2004: 19). Some of them might not want to be locked in under international law. Therefore, ratcheting might be complicated to introduce at either positive or negative list schedules.

\smallskip

The latter Japanese EPAs exemplify the possibility of combining the characteristics of both approaches. Sometimes the previous characteristics are not a defining feature of the negative/positive list, and negotiating parties can just add to their agreement, whatever its schedule technique is. Nowadays, through a hybrid approach to scheduling, parties can adapt their preferences and create treaties closer to their liberalization wishes.

%% TABLE

\section{The empirical strategy}

As can be seen from the previous paragraphs, this design issue has garnered attention but there is no clear conclusion about the differences in terms of liberalization reached. So far, few articles tackle the debate from an empirical perspective.

\smallskip

Up until now, there have been several approaches to compare both listing methods. For example, Broude and Moses (2015) addressed the implications of the positive/negative structure looking at the negotiator’s choices and their subsequent decision-making. They used behavioral law and economics concepts to analyze the dynamics of service trade liberalization and their negotiations. Previously, there have been other empirical attempts to compare negative list and positive list in terms of the extent of liberalization reached by the agreement. All of them have been based on particular methods of scoring and indexing. For example, Dee et al. (2007) created their own system of scores for 12 Free Trade Agreements, to review their treatment of services in terms of the degree of liberalization reached. Their main conclusion was that the agreements based on a negative-list approach are not always more liberal than the positive list based ones for modes of supply 1 and 2 (2007: 7). They specified that the more critical factor characterizing the effectiveness of liberalization measures are the sectoral reservations, rather than the agreements’ format (2007: 29). In 2006, Roy, Marchetti and Lim also started a project to assess and compare liberalization commitments of RTAs.\footnote{Martin Roy and Juan Marchetti continued to work on this dataset (2011) and it is still available at
\url{https://www.wto.org/english/tratop_e/serv_e/dataset_e/dataset_e.htm}.} Even if their article started with a clear differentiation between the NAFTA and the GATS structure, the authors did not go further in their analysis. A relevant conclusion from their study is that countries which have used negative listing for scheduling tend to bound their restrictions at the status quo level. Therefore, negative listing is advisable since members do not spend time negotiating away the margin between commitments and their national applied regime (2006: 53).

\smallskip

In a more concrete study, Adlung and Mamdouh (2013) selected 40 agreements and measured their commitments, thereby completely comparing both schedules techniques. Based on the calculations done by Miroudot et al. (2010), the authors compare sector-specific departures from GATS schedules in 19 bottom-up scheduling agreements versus 21 top-down scheduling agreements. Adlung and Mamdouh conclude that, at first sight, it seems that the negative list agreements reach more ambitious bindings. However, negative list agreements contained more provisions that are less favorable than the corresponding countries’ GATS commitment. The authors highlight that “the underlying scheduling approaches may have been chosen independently of an intended – more or less ambitious- outcome” (Adlung and Mamdouh 2013: 17). What actually matters is the negotiators’ strength of purpose (as the authors refer to, “the political impetus”) and not their preference for a particular architectural approach. It seems that, at a first glance, scoring solutions do not provide a clear statement about which scheduling technique reaches higher liberalization effects.

\smallskip

Finally, Fink and Molinuevo (2007) analyzed this question through an econometric evaluation for the first time. The authors wanted to discern the effect of the scheduling approach, contributing to the debate of whether a negative list approach incentivizes the scheduling of more liberal commitments. In the first place, Fink and Molinuevo created a database of 25 East Asian free trade agreements, where they scrutinized every commitment. They identified the “value added” of these FTAs for each of the 154 sub-sectors and four modes of supply classified under the GATS. Afterwards, they divided the 615 entries per FTA schedule into four different categories: the ones which existed in GATS but the FTA does not improve (GATS only); the commitments which existed in GATS and the FTA eliminated one or more trade-restrictive measure (FTA improvement); commitments which did not exist in GATS but are undertaken at the FTA level (FTA new sectors); and then the ones where neither a GATS nor a FTA commitment exists (Unbound). For these four categories, they further distinguished between partial and full commitments, with the latter appearing when there is no remaining trade-restrictive measure (Fink and Molinuevo 2007: 86). With this classification, the authors compared the aggregate liberalization content by scheduling approach. Their conclusion was that “the group of negative list agreements had unleashed a greater share of improved or new commitments (60 percent) than positive list agreements (30 percent)” (Fink and Molinuevo 2007: 62). The authors continued with an econometric investigation, through which they attempted to control for countries’ propensity to commit to open service markets. They used as independent variables the FTA partners’ level of economic development, the number of FTA parties and the parties’ prior GATS commitments. Also as an independent variable, they introduced their key point of their investigation: a dummy variable that took the value of 1 if the commitment was scheduled under a negative list and 0 otherwise. The dependent variable was based on the previously explained FTAs’ commitments classification, but only divided between “Improved FTA”, “New FTA commitment” and their sum at “Total FTA contribution”. Their findings suggest that “the negative listing does not hold any advantage in deepening GATS commitments, but that they induce the scheduling of new sub-sectors and modes of supply” (p. 64), “in other words, negative listing appears to bring about wider but not deeper commitments” (p. 63). In accordance with Dee et al. (2007) results, Fink and Molinuevo explain that the main distinguishable characteristic of their sample’s negative list agreements is the way sectors are listed; therefore this seems to have an effect on the negotiating outcome. However, Fink and Molinuevo caution against the interpretation of their econometric results, since their database does not capture the depth of the FTA commitments and is just based on their value-added with respect to GATS. Moreover, they add that propensity of countries to commit to open service markets is imperfectly captured with just the four control variables considered at the model. It is difficult to state that the model controls for reverse causality, in the sense that since countries willing to bind liberal trade policies are more likely to adopt a negative list agreement.

\smallskip

Even if the pro-liberalization benefits of the negative list seem to be more predominant, there is no straightforward relationship between the scheduling style and the liberalization ambition of that agreement. The empirical studies depicted above give a glimpse of which are the liberalization contents and extents by scheduling approach. In this section, following the steps of Fink and Molinuevo (2007), the liberalization content of RTAs with different scheduling approaches is compared in quantitative terms. This leads to the question whether a RTA’s approach towards scheduling commitments matters. Formulated as a hypothesis

\begin{itemize}
    \item[$H_1$:] RTAs with negative list schedules reach a higher level of liberalization.
\end{itemize}

It is obvious that the transparency, credibility and regulatory quality of a RTA rely partly on the nature of its trade-restrictive measures scheduled. Knowing the impossibility to measure these characteristics in a quantitative approach, this method is based on the liberalization degree reached by every commitment.

\subsection{The dependent variable}

The dependent variable is a score that tries to capture the liberalization reach by every commitment, on a country level. It builds on a previous analysis of services schedules done for 56 RTAs in which an OECD country is a party. The authors of this study (Miroudot et al. 2010) assessed the preferential content of these RTAs through an analysis of market access and national treatment at the level of 155 sub-sectors of the GATS Sectoral Classification List. The procedure is as follows.

\smallskip

All the commitments are interpreted individually, divided by the country whose schedule of commitments is analyzed, plus the mode of supply and the specific sector and sub-sector according to the W/120 classification (the services sectoral classification list covered by GATS). Once this is done, Miroudot et al. distinguish what kind of non-conforming measure is depicted by the commitment. In addition, restrictions for subsectors and mode are also classified according to whether they pertain to the discipline of market access or national treatment. As Miroudot explained in a later article “in doing so the nature of the restriction itself is also considered, i.e., whether these restrictions are licensing requirement, residency requirements, discriminatory measures regarding taxes and subsidies, restrictions on foreign ownership,(…)” (van der Marel and Miroudot 2014: 214). Once this classification is done, an initial score of 100 is assigned to each services RTA, country, sub-sector and mode of supply regardless of its degree of commitments. Depending on whether the sub-sector is fully, partially or non-committed, an amount of points are deducted from the initial score of 100 according to the restriction and mode. In notation this means

\begin{equation}
    \text{RTA}^s_{od} = 100 - X^s_{od}
\end{equation}

for each sector $s$ and mode, commitment undertaken by country $o$ towards partner country $d$. Whereas $X^s_{od}$ indicates the minus score assigned to the RTA commitment according to the type of restriction. This scoring system follows the pattern done by van der Marel and Miroudot (2014). They assumed that market access matters are relatively more restrictive than national treatment. Their reasoning is that “entry barriers and other quantitative restrictions in services are more trade-restrictive than discriminations between foreign and domestic firms” (van der Marel and Miroudot 2014: 214). Therefore $X$ can take the following values, depending on the commitments’ nature.

%% TABLE

This means that full commitments or completely open sectors will take a score of 100. Then the methodology will rank the importance of the partial commitment, thus showing how restrictive to foreign services a RTA is. Unbound subsectors end with a score of 20 points. Furthermore, the four different modes are weighted inside every sector. The weighting scheme for the index assigns a weight of 41.2\% to Mode 1 and 3 and only 15.5\% and 2.1\% to Mode 2 and 4 respectively. The authors based their weighting scheme on the share value of trade in services by mode of supply estimated in Hoekman and Kostecki (2009).

\smallskip

This method gives an accurate score to every commitment done by every country participant in these 56 RTAs. It reflects full commitments, partial commitments (counting to which degree that commitment opens the domestic market) and it also reflects unbound commitments. Of course, as Miroudot et al (2010: 10) already stated, this methodology also presents its own caveats. It does not separate between de jure and de facto trade regimes. Perhaps the legal bindings present at the RTA level are not portraying the actual trade regime at the country level. Or, even if a treaty shows great potential for liberalization of trade, it might not imply increased bilateral trade flows between both countries. Since the present study’s purpose is to analyze differences in treaties’ legal clauses, this issue could be overlooked. However, Miroudot et al (2010: 10) also warn about possible errors and misinterpretations when constructing the database, given the numerous commitments reviewed. For example, some countries use a different sector classification than the W/120 used and it was necessary to check its CPC correspondence. The same applies with the modes, since some negative lists do not explicitly say in which mode the measure is implemented. The analysis then needed some interpretation efforts that might lead to an “approximation” of what some existing commitments intended.

\subsection{Model specification and data sources}

The empirical strategy takes as a first step the econometric evaluation introduced by Fink and Molinuevo (2007). It is then developed with more independent variables and a different and broader sample of RTAs. The dependent variable is also different, since it is used the previously explained index from van der Marel and Miroudot (2014). However, instead of their 56 RTAs analyzed, this regression only took 50 of them, where 25 are positive and 25 are negative bilateral trade in services treaties. The oldest treaty in the sample entered into force in 1995 (Mexico-Costa Rica FTA), the newest in 2011 (EU-Korea FTA). The 6 missing treaties was due to the dropping of multilateral treaties present, in order to avoid possible free riding effects. Also, treaties needed to have a clear distinction between the positive and negative listing approach. Therefore, the final sample stayed with 50 treaties instead on the 56 analyzed by van der Marel and Miroudot (2014).

\smallskip

This is a cross-sectional study. One of the main challenges is that the values of this index ranged from 20 until 100 points. The distribution of the RTAs index variable is U-shaped, due to the fact that unbound and full commitments sectors are more common (20 and 100 points are more repeated than partial commitments). Due to this nature, the index was transferred into a 0 until 1 scale. As this dependent variable is an index, defined only in the interval $[0,1]$, it is tried a new approach different from their OLS model: a fractional regression model (Papke and Wooldridge 1996). The tobit option was ruled out due to the fact that 0 and 1 are a consequence of the index’s nature and not censoring. The data was censored in this interval and it was searched a model with application to data defined only in that interval.

\smallskip

Developed by Papke and Wooldridge (1996), this model deals with dependent variables defined on the closed interval $[0,1]$. In this model, as Ramalho and Silva (2009) describe, Papke and Wooldridge (1996) assume a functional form for $Y$ that imposes the desired constraints on the values of the dependent variable:

\begin{equation}
    E(Y|X=G(X\beta)
\end{equation}

where $G(\cdot)$ is a known nonlinear function satisfying $0 < G(\cdot) < 1$ with as possible specification any cumulative distribution function, like in this case the logic function. $Y$ is the dependent variable (the RTA index), $X$ denotes a matrix containing all explanatory variables that are going to be referred in the next section and $\beta$ is the vector of variable coefficients that we aim to estimate. This model is estimated directly using nonlinear techniques. Papke and Wooldridge (1996) showed it is more efficient to assume a Bernoulli distribution for $Y$ conditional on $X$ and estimate the parameters in $\beta$ by maximizing the quasi-likelihood function.

\subsection{Independent variables}

At the independent variable level, this specification takes a strong influence from several gravity models in trade in services. First, it includes data on real GDP and population\footnote{GDP, population and distance varibles are included with their natural-log transformation.} from both the country specifying the commitment and the partner country at the bilateral treaty, and both at the entry into force year; data comes from the World Bank’s World Development Indicators. Then it includes geographical variables (distance and common border between both countries) and a common language variable, from CEPII. It also includes a GATS index variable for the country making the commitment, equally calculated by Miroudot et al (2010). It is meant to show the countries propensity to open their services markets to international trade. Also, to control for the country’s economy relative importance to services trade, it is included the percentage that trade in services represent in their GDP for the entry into force year, as the sum of service exports and imports divided by the value of the country’s GDP for that year (also from the World Bank’s World Development Indicators). It is also included a Democracy variable, sourced from Freedom House aggregate scores. The most important variable is the dummy $\textit{neglist}_o$ the value 1 if the commitment was scheduled under a negative list and 0 otherwise (self-constructed). Finally, it includes fixed effects for the 12 service sectors present at the commitments classification.

\subsection{Results}

%% TABLE

As Ramalho and Silva (2009: 17) explain, this fractional regression uses a Bernoulli distribution for $Y$ conditional on $X$ and to estimate the parameters $\beta$. Since the Bernoulli distribution is a member of the linear exponential family, “the resulting quasi-maximum likelihood (QML) estimator for $\beta$ will always be consistent, regardless of the true distribution of $Y$ conditional on $X$, provided that the functional form is correctly specified”\footnote{As suggested by Ramalho and Silva, it was used the RESET test to check the hypothesis of correct functional form specification. Results suggested that this functional form used by the FRM is not misspecified. The RESET results will be included in the final version of this article.} (Ramalho and Silva 2009: 17). Following Ramalho and Silva, as in no circumstances can the Bernoulli be the true conditional distribution of the dependent variable (in this case the RTAs index commitments), robust standard errors have to be used.\footnote{The command “fracreg” used by Stata for the fractional regression models computes robust standard errors by default.} To better interpret the results, we can see the average marginal effects (AME).

%% TABLE

This reports the AME, where we could see the change in the outcome variable for a change in the independent ones. Thus, being a treaty categorized as negative list relative to positive list increases on average the RTA index (liberalization at the commitments level) by 31\%, holding the exact same values on the other independent variables in the model. The estimates obtained for the model indicated that most of the variables have a significant influence in the RTA index, excluding the percentage that trade in services represent in the partner/destination country (the one not subscribing the commitments). The positive and negative relation between the RTA index and the explanatory variables also seem rational. For example, the negative effect of the population at both countries in the treaty might make no sense for a gravity model in trade in goods. However, in trade in services, small countries at the sample have higher relative importance or weight of their trade in services flow in their GDP than bigger countries. They also tend to sign more trade in services agreements. See, for example, the case of Singapore or Costa Rica.

\smallskip

These model results cannot be used for inferences, but we could obtain estimates for the very first time for the sample treaties. With the OLS regression, causality not only runs from the scheduling approach to the liberalization outcome, but also the other way around. Since usually the negotiators at a country level choose the scheduling technique, they might prefer a concrete scheduling technique depending on their liberalization purpose. As Fink and Molinuevo (2007: 20) already stated, trading partners that are cautious in committing to market opening in services may be more likely to adopt a positive list. Or the other way round; countries that have a good understanding of all their trade in services measures and that are prepared to open up their markets are more likely to adopt a negative list agreement. Therefore, reverse causality bias represents a risk in this model.

\section{Conclusion}

While the share of services value added in GDP increases in both developed and developing countries and trade in services grows its relative importance inside global production and world trade; international treaties with services components remain underinvestigated. This article tried to provide more insights about one concrete aspect of trade in services agreement design. Since the mid-nineties, NAFTA and GATS started two opposite styles in the manner that a given government depicts the list of services sectors subject to the treaty obligations. With the positive list, a negotiating party does not make a commitment to liberalize until explicitly inscribe it in their schedule of commitments, and to the extent of the inscription. With the negative list, all services sectors are fully committed by the negotiating parties unless explicitly excluded, and to the extent of the exclusion. When comparing both styles, the depiction of sectors, disciplines and modes are different. In the positive list, we talk about schedules of commitments. In the negative list, we tend to see annexes of reservations.

\smallskip

As it has been argued before, the negative list approach presents a series of features that might provoke a dynamic of negotiation which reaches higher levels of trade openness. However, as it was also discussed, this claim has not been undisputed among trade in services scholars. The article tried to provide one of the most complete compilations of negative list advantages claimed so far by experts, and at the same it portrayed the possible pro-liberalization features of the positive list.

\smallskip

Finally, the article displayed an econometric study which tried to provide an empirical evaluation of the negative lists’ superiority for reaching deeper liberalization commitments. So far, to the knowledge of this author, there has been only a previous case where an econometric investigation was used to ascertain the influence of the negative list. However, the authors recognized the limitations of their model: it did not capture the true depth of FTA commitments, only value-added relative to the GATS and only for 25 East-Asia FTA’s agreements. Changing their definition of the dependent variable, this article used an innovative index created by Miroudot et al. to measure the trade in services openness and its depth content for 50 recent trade in services treaties (25 bilateral positive listing and 25 bilateral negative listing). The independent variables tried to control for the characteristics of both negotiating parties and were strongly inspired by the gravity model. Of course, as a main variable of interest, it was introduced a dummy variable to differentiate between positive and negative list commitments in the sample. However, due to the nature of the dependent variable, which ranges from 0 until 1, it was used a fractional regression model. The results showed strong support for the hypothesis that the negative list means a significant higher level of liberalization.

\smallskip

For future steps, it remains to be seen if the negative list means not only a higher trade-openness at the treaty commitments’ level (de iure effects), but if it also channels higher bilateral trade flows between both signatories when compared with positive list agreements (de facto effects).

\appendix

%% Table

\end{document}
